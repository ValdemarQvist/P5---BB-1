\part{Problem analysis}\label{Part:PA}
\chapter{Introduction}\label{ch:introduction}
Many traditional daily activities, like grocery shopping or home depot shopping, have entered the world of e-commerce. A large number of companies offer services that will bring you, your self-chosen groceries straight to your door step. This may seem like an improvement of the old-fashioned way of going to a store. However, one of the issues with online grocery shopping, is the ability to see, feel and smell the goods, and this goes for especially the fresh items like fruits and vegetables \cite{GroceryDive}.\\

Customers are steadily getting accustomed to the idea of purchasing online, and not only non-foods. This means that the stores have been converting more of their sales from the physical stores to the online version for the past couple of years \cite{Quartz}. This could indicate that customers may be ready to let go of the traditional shopping method in exchange for extra time on their hands.\\
\\
Shopping online offers the possibility for knowing whether an item is available or not as well as makes finding the items in the store easier whereas this is not as effortless when going to physical store. This is due to it requiring time and energy and asking for help may not always be as easy as it may sound. F.Lee (2002) \cite{AskingHelp} discusses three problematic aspects of asking other people for help. The first is accepting lack of competence to complete a task. By asking for help, you acknowledge that you can not finish a task, either in time or at all by yourself. Secondly, it involves accepting inferiority to other people. By asking them, you agree that you think they may be better equipped to complete this task than yourself. Lastly, it involves accepting a level of dependency of other people. A robot helper might not entirely remove these factors, as the person would still have to accept these things, but removing the human interaction might alleviate some of the pressure and encourage customers to ask the robot for help.\\

Pros and cons being presented for both online and traditional shopping, a middle way could be a way of utilising the pros of each method for shopping. A new method could be introduced that would reduce the time spent shopping compared to the traditional way. This method would still allow self inspection of wares and lowering or removing the anxiety related to asking employees for help, while lowering the running costs of a store, by lowering the amount of employees needed for costumer service.

\newpage
The case for this will be presented as follows:\\

\textit{A robot that can guide customers around in hypermarkets (e.g. Bilka, Silvan, Target, Walmart) to reduce time spent searching for items and reduces/eliminates the need for employees to be present for customers in need of assistance with locating wares. This means that the running costs of the store could potentially be lowered and the time consumption is reduced for the customers, while still giving the customers the choice to choose their goods from the shelves. This can be coupled with the company's online shop, to allow for a pre-built shopping list with immediate guidance upon arrival or the robot can offer a look at the stores full selection of items via a monitor. The robot is meant to be part of a fleet that would operate in collaboration with each other in the warehouse, as a store will typically have more than one customer at a time.}\\

Designing a robot as described above requires many problems solved. This includes: navigation, path planning, interaction with surroundings and social interaction.\\

\textbf{Navigation} is a crucial part of a solution that revolves around guiding people to specific locations. If the robot cannot locate itself, it cannot complete a given task. Navigating around a store includes mapping both static and dynamic objects.\\

\textbf{Path planning} is a part of navigation, in the sense that it is also important to deal with the robots decision making program, as not all paths may be equally time efficient. This will also be the part of the navigation that deals with dynamic objects and obstacles.\\

\textbf{Interaction with surroundings} revolves around the use of sensors to give the robot the ability to react to its surroundings. This can include pictures/video, range tracker and software to translate the sensor data into data that can be processed and yield information about proximity of objects, whether they are moving or not (depending on the frequency of the system).\\

\textbf{Social interaction} is inevitably a part of designing a robot that is intended to operate within the same boundaries as humans. This will include giving the user a good experience based on the robots behaviour. It is important for humans to feel safe and comfortable around the robot for the best customer experience, including best management of time spent at the store.\\